\documentclass[12pt,a4paper]{article}
\usepackage[utf8]{inputenc}
\usepackage[T1]{fontenc}
\usepackage[french]{babel}
\usepackage{geometry}
\geometry{margin=2.5cm}
\usepackage{graphicx}
\usepackage{color}
\usepackage{tcolorbox}
\usepackage{enumitem}
\usepackage{titlesec}
\usepackage{setspace}
\usepackage{hyperref}

\hypersetup{
    colorlinks=true,
    linkcolor=blue,
    urlcolor=blue
}

\setstretch{1.2}
\setlist[itemize]{topsep=3pt,itemsep=3pt,parsep=0pt}

\title{\textbf{Problèmes des Startups et Définition de la Proposition de Valeur}}
\author{---}
\date{}

\begin{document}

\maketitle
\tableofcontents
\newpage

\section*{Introduction}
\addcontentsline{toc}{section}{Introduction}
\textbf{Q :} Problème de 90\% des startups \\
\textbf{R :} Le problème, c'est le problème que la solution résout.

\section{Les 6 erreurs à éviter}

\subsection{I. Problème mal défini ou inexistant}
\textit{“Solution in Search of a Problem” (SISP)}  
L’équipe part d’une idée, d’une technologie ou d’une fonctionnalité innovante sans avoir identifié un besoin réel.

\begin{tcolorbox}[colback=blue!5,colframe=blue!50!black,title=Exemple]
Application de réalité augmentée pour cartes de visite sans demande manifeste.
\end{tcolorbox}

\subsubsection*{Problème non prioritaire ou insuffisamment douloureux}
Problématique réelle mais non urgente ou peu importante pour les utilisateurs.

\begin{tcolorbox}[colback=blue!5,colframe=blue!50!black,title=Exemple]
Affirmer en 2000 que les motocyclistes béninois se sentent en danger sans casques.
\end{tcolorbox}

\subsubsection*{Problème trop général ou mal formulé}
La définition est floue, trop large ou non actionnable.

\begin{tcolorbox}[colback=blue!5,colframe=blue!50!black,title=Exemple]
“Les gens veulent mieux gérer leur temps.”
\end{tcolorbox}

\paragraph{Pour une plateforme EdTech :}
Préciser le problème réel :  
\begin{itemize}
    \item Manque de matériel dans les zones rurales ?  
    \item Accès limité aux ressources de préparation aux examens ?  
\end{itemize}

\subsection{II. Biais cognitifs du fondateur}
\subsubsection*{Biais de projection personnelle}
Le fondateur généralise son propre cas.

\begin{tcolorbox}[colback=blue!5,colframe=blue!50!black,title=Exemple]
Développer une application d’organisation détaillée car on en ressent soi-même le besoin.
\end{tcolorbox}

\subsubsection*{Biais de confirmation}
Les retours positifs d’un cercle proche sont interprétés comme validation du marché.

\begin{tcolorbox}[colback=blue!5,colframe=blue!50!black,title=Exemple]
“Mes amis trouvent l’idée géniale.”
\end{tcolorbox}

\subsubsection*{Biais d’optimisme}
Sous-estimation de la difficulté d’adoption.

\begin{tcolorbox}[colback=blue!5,colframe=blue!50!black,title=Exemple]
Google a arrêté depuis 2006 plus de 160 produits comme Google+, Apple Maps a été un échec initial.
\end{tcolorbox}

\subsection{III. Problème mal positionné dans le cycle utilisateur}
\subsubsection*{Problème trop en amont}
Le besoin existe mais n’est pas encore perçu.

\begin{tcolorbox}[colback=blue!5,colframe=blue!50!black,title=Exemple]
Assurances habitation au Bénin vs France.
\end{tcolorbox}

\subsubsection*{Problème secondaire ou dérivé}
Cibler un symptôme plutôt que la cause racine.

\begin{tcolorbox}[colback=blue!5,colframe=blue!50!black,title=Exemple]
Proposer des outils de motivation sans résoudre le manque d’accès aux contenus pédagogiques.
\end{tcolorbox}

\subsection{IV. Problème mal segmenté}
\subsubsection*{Ciblage trop large}
S’adresser à tout le monde rend la proposition floue.

\begin{tcolorbox}[colback=blue!5,colframe=blue!50!black,title=Exemple]
“Tout le monde veut manger plus sainement.”
\end{tcolorbox}

\subsubsection*{Problème réel mais cible non solvable}
Public mal équipé, peu digitalisé ou sans moyens.

\begin{tcolorbox}[colback=blue!5,colframe=blue!50!black,title=Exemple]
Proposer un outil SaaS à des agriculteurs sans connectivité fiable.
\end{tcolorbox}

\subsection{V. Problème structurellement complexe (“Tarpit Ideas”)}
\subsubsection*{Complexité opérationnelle}
Obstacles logistiques, réglementaires ou financiers majeurs.

\begin{tcolorbox}[colback=blue!5,colframe=blue!50!black,title=Exemple]
Développer une école low-cost avec enseignement d’excellente qualité.
\end{tcolorbox}

\subsubsection*{Secteur lent à évoluer}
\begin{tcolorbox}[colback=blue!5,colframe=blue!50!black,title=Exemple]
Solutions pour l’éducation publique ou secteur hospitalier.
\end{tcolorbox}

\subsection{VI. Validation insuffisante ou biaisée}
\begin{itemize}
    \item Absence de validation terrain  
    \item Validation qualitative surinterprétée  
    \item Confusion entre intention et action  
\end{itemize}

\newpage
\section{Exemple de bonne définition de problème}

\begin{tcolorbox}[colback=gray!5,colframe=black,title=Comment bien définir un problème]
\begin{itemize}
    \item Personnes ou entités concernées clairement identifiées  
    \item Pain point clair  
    \item Contexte géographique et temporel défini  
    \item Conséquences néfastes et importantes  
    \item Aucune solution satisfaisante existante  
    \item Explication claire du pourquoi  
\end{itemize}
\end{tcolorbox}

\paragraph{Exemple illustratif :}
\begin{quote}
2 millions de personnes n’ont pas accès à l’eau potable en zones rurales au Bénin, réduisant de 10 ans leur espérance de vie.  
Elles doivent se contenter de forages de fortune car l’État ne peut financer l’extension du réseau dans ces zones à faible densité.
\end{quote}

\section{Différence entre Solution et Produit}
\textbf{La solution n’est pas le produit.}  
C’est la manière dont vous résolvez le problème.  
Une solution peut se décliner en plusieurs produits.

\subsubsection*{Exemples}
\textbf{Netflix}  
\begin{itemize}
    \item \textbf{Problème :} Les clients payaient pour chaque film.  
    \item \textbf{Solution :} Catalogue illimité à prix fixe.  
    \item \textbf{Produit :} Plateforme de streaming.
\end{itemize}

\textbf{Airbnb}  
\begin{itemize}
    \item \textbf{Problème :} Les hôtels coûtent cher et manquent d’authenticité.  
    \item \textbf{Solution :} Logement chez l’habitant sécurisé et abordable.  
    \item \textbf{Produit :} Plateforme de réservation entre particuliers.
\end{itemize}

\section{Proposition de valeur}
C’est la promesse faite au client, ce qu’il gagne à utiliser votre service.

\textbf{Exemples :}
\begin{itemize}
    \item \textbf{Netflix :} Accès illimité, instantané et personnalisé à un large catalogue, à un tarif abordable et sans publicité.
    \item \textbf{Airbnb :} Se sentir comme chez soi partout dans le monde.
\end{itemize}

\section{Validation de marché}
Cinq moyens de valider un marché :
\begin{enumerate}
    \item Identifier des concurrents directs (ex : WAVE, Orange Money)
    \item Trouver des \textit{proxies} (ex : Netflix, Spotify)
    \item Identifier un acteur ayant réussi à l’étranger (ex : Yango, Gozem)
    \item Préempter un marché existant (ex : Wetaxi, Yango)
    \item Enregistrer un intérêt fort (ex : Dropbox : 75 000 inscrits en 24h)
\end{enumerate}

\section{TAM, SAM, SOM}
\textbf{TAM (Total Addressable Market)} : Marché total adressable.  
\textbf{SAM (Serviceable Available Market)} : Marché atteignable aujourd’hui.  
\textbf{SOM (Serviceable Obtainable Market)} : Marché réellement transformable.

\section{Analyse de la concurrence}
\begin{itemize}
    \item Identifier des concurrents avec les mêmes besoins, profils clients et géographies.  
    \item Lister et segmenter les concurrents.  
    \item Identifier leurs points forts et vos avantages comparatifs.  
    \item Déterminer votre “\textit{secret sauce}”.  
\end{itemize}

\section{Vision et orientation stratégique}
\textbf{Exemple : Amazon}  
Vision : « Être l’entreprise la plus centrée sur le client au monde. »

\subsubsection*{Résultats :}
\begin{itemize}
    \item De la vente de livres à l’e-commerce global  
    \item Création d’AWS  
    \item Diversification dans le streaming et la logistique
\end{itemize}

\subsection*{Comment définir sa vision ?}
\begin{itemize}
    \item \textbf{Clarté :} Phrase courte et compréhensible.  
    \item \textbf{Inspiration :} Motive et attire les talents.  
    \item \textbf{Durabilité :} Survit aux pivots.  
    \item \textbf{Impact :} Exprime la valeur créée.  
\end{itemize}

\vspace{1em}
\noindent\textit{N’hésite pas à poser des questions si certains points te paraissent flous.}

\end{document}
